% Modelo de TCC do Bacharelado em Ciência da Computação da UNIFESP 
% Baseado no Modelo de Documentos Academicos do ABNTex2  

\documentclass[	12pt, Times, openright, twoside, a4paper, english, brazil]{abntex2}

% ---
% Pacotes fundamentais 
% ---
\usepackage{cmap}				% Mapear caracteres especiais no PDF
%\usepackage{lmodern}			% Usa a fonte Latin Modern			
\usepackage{times}
\usepackage[T1]{fontenc}			% Selecao de codigos de fonte.
\usepackage[utf8]{inputenc}		% Codificacao do documento (conversão automática dos acentos)
\usepackage{lastpage}			% Usado pela Ficha catalográfica
%\usepackage{natbib}
\usepackage{indentfirst}			% Indenta o primeiro parágrafo de cada seção.
\usepackage{color}				% Controle das cores
\usepackage{graphicx}			% Inclusão de gráficos
% ---
\usepackage[portuguese, ruled, linesnumbered]{algorithm2e} %Peseudocodigo
\usepackage{amssymb} %checkmarker
% ---
% Pacotes de citações
% ---
\usepackage[brazilian,hyperpageref]{backref}	 % Paginas com as citações na bibl
\usepackage[alf]{abntex2cite}	% Citações padrão ABNT

% --- 
% CONFIGURAÇÕES DE PACOTES
% --- 

% ---
% Configurações do pacote backref
% Usado sem a opção hyperpageref de backref
\renewcommand{\backrefpagesname}{Citado na(s) página(s):~}
% Texto padrão antes do número das páginas
\renewcommand{\backref}{}
% Define os textos da citação
\renewcommand*{\backrefalt}[4]{
	\ifcase #1 %
		Nenhuma citação no texto.%
	\or
		Citado na página #2.%
	\else
		Citado #1 vezes nas páginas #2.%
	\fi}%
% ---

% numeração de figuras e elas 
\counterwithout{figure}{section}
\counterwithout{table}{section}

%\renewcommand\tablename{Tabela{\arabic{chapter}.}}


% ---
% Informações de dados para CAPA e FOLHA DE ROSTO
% ---
\titulo{ANÁLISE DE DEMANDA VIA INTELIGÊNCIA ARTIFICIAL NO RESTAURANTE UNIVERSITÁRIO DO INSTITUTO CENTRAL DE TECNOLOGIA DA UNIFESP}
\autor{Douglas Diniz Landim}
\local{São José dos Campos, SP}
\data{Março de 2018}
\orientador{Prof. Dr. Vinicius Veloso}
%\coorientador{Prof. Dr. }
\instituicao{%
  Universidade Federal de São Paulo -- UNIFESP
  \par
  Instituto de Ciência de Tecnologia
  \par
  Bacharelado em Ciência da Computação}
\tipotrabalho{Trabalho de Graduação}
% O preambulo deve conter o tipo do trabalho, o objetivo, 
% o nome da instituição e a área de concentração 
\preambulo{Trabalho de conclusão de curso apresentado ao Instituto de Ciência e Tecnologia – UNIFESP, como parte das atividades para obtenção do título de Bacharel em Ciência da Computação.}
% ---

% informações do PDF
\makeatletter
\hypersetup{
     	%pagebackref=true,
		pdftitle={\@title}, 
		pdfauthor={\@author},
    	pdfsubject={\imprimirpreambulo},
	    pdfcreator={LaTeX with abnTeX2},
		pdfkeywords={abnt}{latex}{abntex}{abntex2}{trabalho acadêmico}, 
		colorlinks=true,       		% false: boxed links; true: colored links
    	linkcolor=blue,          	% color of internal links
    	citecolor=blue,        		% color of links to bibliography
    	filecolor=magenta,      		% color of file links
		urlcolor=blue,
		bookmarksdepth=4
}

\makeatother
% --- 
% --- 
% Espaçamentos entre linhas e parágrafos 
% --- 
% O tamanho do parágrafo é dado por:
\setlength{\parindent}{1.3cm}
% Controle do espaçamento entre um parágrafo e outro:
\setlength{\parskip}{0.2cm}  % tente também \onelineskip
% ---

% compila o indice
% ---
\makeindex
% ---

% ----
% Início do documento
% ----
\begin{document}
% Retira espaço extra obsoleto entre as frases.
\frenchspacing 

% ----------------------------------------------------------
% ELEMENTOS PRÉ-TEXTUAIS
% ----------------------------------------------------------
% \pretextual

% ---
% Capa
% ---
\begin{capa}
  \begin{center}
   \includegraphics[width=.25\textwidth]{logo-unifesp.pdf}
    \vspace*{\fill}
    
    {\ABNTEXchapterfont\large\imprimirautor}
    \vspace*{\fill}
    
    {\ABNTEXchapterfont\bfseries\Large\imprimirtitulo}
    \vspace*{\fill}\vspace*{\fill}
    
   \imprimirlocal
   \end{center}
\end{capa}

% ---
% Folha de rosto
% (o * indica que haverá a ficha bibliográfica)
% ---
\imprimirfolhaderosto*
% ---

% ---
% Inserir folha de aprovação
% ---
% Isto é um exemplo de Folha de aprovação, elemento obrigatório da NBR
% 14724/2011 (seção 4.2.1.3). Você pode utilizar este modelo até a aprovação
% do trabalho. Após isso, substitua todo o conteúdo deste arquivo por uma
% imagem da página assinada pela banca com o comando abaixo:
%
% \includepdf{folhadeaprovacao_final.pdf}
%
\begin{folhadeaprovacao}
  \begin{center}
    {\ABNTEXchapterfont\large\imprimirautor}

    \vspace*{\fill}\vspace*{\fill}
    {\ABNTEXchapterfont\bfseries\Large\imprimirtitulo}
    \vspace*{\fill}
    
    \hspace{.45\textwidth}
    \begin{minipage}{.5\textwidth}
        \imprimirpreambulo
    \end{minipage}%
    \vspace*{\fill}
   \end{center}
    
   Trabalho para apresentar em NOVEMBRO/2018:

   \assinatura{\textbf{\imprimirorientador} \\ Orientador} 
   \assinatura{\textbf{Professor} \\ Convidado 1}
   \assinatura{\textbf{Professor} \\ Convidado 2}
   \assinatura{\textbf{Professor} \\ Convidado 3}
   %\assinatura{\textbf{Professor} \\ Convidado 4}
      
   \begin{center}
    \vspace*{0.5cm}
    {\large\imprimirlocal}
    \par
    {\large\imprimirdata}
    \vspace*{1cm}
  \end{center}
  
\end{folhadeaprovacao}
% ---

% ---
% Dedicatória
% ---
\begin{dedicatoria}
   \vspace*{\fill}
   \centering
   \noindent
   \textit{ Este trabalho é dedicado aos meus pais que apoiaram e sacrificaram esforços para me manter ativo nessa jornada, a todos os professores que me somaram conhecimentos, oportunidades e esperanças indo além de suas rotinas e agendas em prol do ensino, e principalmente à todos que me motivaram me oferecendo desafios para que eu pudesse enfrentá-los superando meus próprios limites } \vspace*{\fill}
\end{dedicatoria}
% ---

% ---
% Agradecimentos
% ---
\begin{agradecimentos}
Minha jornada pela graduação foi marcada por muita persistência, dificuldades e fracassos. Agradeço primeiramente a Deus por me dar fé e alimentar minha persistência e esperança. Apesar de todo o conteúdo técnico das mais de 40 disciplinas do meu curso, o que mais me agregou evolução foi o ambiente extremamente desafiador desta universidade; que somado à muitas dificuldades pessoais, acidentes, contra-tempos de saúde, profissão e família; constituiu a soma perfeita de desafios que me derrubaram muitas e muitas vezes e me fizeram ser uma pessoa melhor, mais convicta e mais perseverante a cada nova tentativa de conquistar minhas aprovações. Agradeço à minha família por sempre me apoiar dando tudo de si, a meu professor por me aceitar orientar, e me motivar sempre me dando atenção como um amigo nas conversas no fim da aula durante o caminho até o estacionamento da faculdade, nas reuniões e chats online até nos finais de semana, aos amigos universitários, e a todos os professores que me acompanharam e ofereceram desafios em todos esses anos na UNIFESP. 

\end{agradecimentos}
% ---

% ---
% Epígrafe
% ---
\begin{epigrafe}
    \vspace*{\fill}
	\begin{flushright}
		\textit{Mesmo desacreditado e ignorado por todos, não posso desistir, pois para mim, vencer é nunca desistir.\\
		(Albert Einstein)}
	\end{flushright}
\end{epigrafe}
% ---

% ---
% RESUMOS
% ---

% resumo em português
\begin{resumo}
O presente trabalho tem como objetivo a comparação com o método de análise da previsão de vendas do restaurante universitário da Unifesp, previamente feita pelo autor deste projeto com aplicação de métodos estatísticos, e atualmente com métodos de aprendizado de máquina. Tal análise por aprendizado de máquina foi apontado como relevante solução no fim da análise estatística do trabalho anterior. A temperatura da região onde se localiza o campus do restaurante foi analisada por recorrência via bootstrap como um fator que exerce impacto sobre as vendas do restaurante em certos períodos do semestre, e neste trabalho de conclusão de curso serão obtidas novas variáveis e um intervalo maior de amostragem na análise da previsão de demanda de refeições do restaurante em um novo modelo com aprendizado de máquina a fim de que seja obtido um modelo de previsão viável para evitar super-projeção de demanda com consequência de desperdício de alimentos, ou subprojeção com consequência de docentes ou discentes sem refeições.
 
 \vspace{\onelineskip}
    
 \noindent
 \textbf{Palavras-chaves}: Redes Neurais Artificiais. Previsão de demanda. Aprendizado de Máquina. Inteligência Artificial. Perceptron Múltiplas camadas. 
 
\end{resumo}

% resumo em inglês
\begin{resumo}[Abstract]
 \begin{otherlanguage*}{english}

The present work has the objective of comparison with the static method of analysis of the demand prediction in UNIFESP university restaurant, previously done by the author of this project, and currently with machine learning methods. Such analysis by machine learning was pointed out as a relevant solution at the end of the statistical analysis of the previous work. The temperature of the region where the restaurant campus is located was analyzed by bootstrap recurrence as a factor impacting the restaurant sales in certain periods of the semester, and in this work of completion of the course will be obtained new variables and a greater range of sampling in the analysis of the forecast of restaurant meal demand in a new model with machine learning in order to obtain a viable prediction model to avoid overprojection of demand as a result of food waste or subprojection with the consequence of teachers or students without meals.

   \vspace{\onelineskip}
 
   \noindent 
   \textbf{Key-words}: Artificial Neural Networks. Demand Prediction. Machine Learning. Artificial intelligence. Perceptron Multiple layers.
 \end{otherlanguage*}
\end{resumo}

% ---
% inserir lista de ilustrações
% ---
\pdfbookmark[0]{\listfigurename}{lof}
\listoffigures*
\cleardoublepage
% ---

% ---
% inserir lista de tabelas
% ---
\pdfbookmark[0]{\listtablename}{lot}
\listoftables*
\cleardoublepage
% ---

% ---
% inserir lista de abreviaturas e siglas
% ---
\begin{siglas}
\item[ICT] Instituto Central de Tecnologia
\item[R.U] Restaurante Universitário
\item[UNIFESP] Universidade Federal de São Paulo
\item[BDMEP] Banco de Dados Meteorológicos para Ensino e Pesquisa

\end{siglas}
% ---

% ---
% inserir lista de símbolos
% ---
%\begin{simbolos}
%  \item[$ \Gamma $] Letra grega Gama
%  \item[$ \Lambda $] Lambda
%  \item[$ \zeta $] Letra grega minúscula zeta
%  \item[$ \in $] Pertence
%\end{simbolos}
% ---

% ---
% inserir o sumario
% ---
\pdfbookmark[0]{\contentsname}{toc}
\tableofcontents*
\cleardoublepage
% ---

% ----------------------------------------------------------
% ELEMENTOS TEXTUAIS
% ----------------------------------------------------------
\textual

% ----------------------------------------------------------
% Introdução
% ----------------------------------------------------------
\chapter{Introdução}
\section{Contextualização e Motivação}

\paragraph*{} A previsão de demanda é um ponto de extrema importância para qualquer empresa, uma previsão adequada permite o ajuste de todo o seu mecanismo de operações para atender tal demanda com a melhor eficiência possível, maximizando lucros, minimizando perdas, e principalmente atendendo todas as necessidades do cliente.

\paragraph*{} Todo restaurante universitário enfrenta problemas de previsão de demanda de refeições e prejuízos com a falta de vendas e ou o descarte de refeições não vendidas. Um dos grandes problemas enfrentados hoje no mundo é a elevação dos preços dos alimentos, o valor além de monetário é moral, a alimentação é o recurso primitivo de base da humanidade que hoje ainda enfrenta um mal acesso a este em muitas regiões carentes. O descarte indevido de alimentos, provocado por suas limitações e durabilidade não gera somente prejuízos monetários e sim prejuízos ambientais e morais. Isso tem causado preocupações para a população em geral e também para empresas como restaurantes que sofrem diretamente os reflexos da variação no preço dos alimentos e na demanda. Atualmente o restaurante universitário do ICT - UNIFESP não possui um sistema que ajude na gestão de compras dos alimentos.

\paragraph*{} No restaurante universitário do ICT UNIFESP as refeições são fornecidas de segunda a sexta feira. O caso particular de restaurantes universitários envolve um fluxo de demanda influenciadopor dia da semana, visto que a demanda é influenciada pela quantia de alunos presentes na universidade, que por sua vez é influenciada pela grade de aulas determinada semestralmente por dia da semana. O caso de análise para este projeto foi motivado após informações de relevantes desperdícios.

Devido às condições burocráticas no ambiente do restaurante que compreendem fidelidade de contrato, exclusividade na região pois o restaurante se encontra em localização que o faça ser o único provedor de alimentos ao público do campus estando isolado fisicamente de qualquer região comercial, e acessibilidade do público à aquisição de refeições que em sua maior parte adquirem refeições pelo valor de 2,50 sendo o restante do custo da refeição subsidiada pela instituição através de verbas federais, a escolha dos parâmetros não será influenciada por muitos fatores externos como concorrência, acessibilidade do ponto, entre outros.

\paragraph*{} Outro ponto importante é a obtenção dos valores de venda; não foram escolhidas as vendas diretas do ponto de venda de tickets de refeição, e sim os dados de coleta da entrada do restaurante, que demonstram a real movimentação de público no restaurante em determinado dia.

\paragraph*{} O estudo da relação de vendas, temperatura, outras variáveis climáticas e do ambiente, já é comum em outros cenários, entre eles o de maior destaque é na demanda de energia elétrica. Os cenários de vendas de alimentos perecíveis ganha também destaque apesar de se encontrar investimentos maiores na indústria de produção de energia elétrica. O objetivo tanto no cenário deste trabalho, o restaurante universitário, como em outros cenários é o mesmo, atender toda a demanda de consumo e evitar transtorno à qualquer consumidor pela falta da mesma, e evitar prejuízos de produção não consumida. Tais prejuízos impactam não só ao fornecedor, mas também ao consumidor, um fornecimento de produto e serviço com um bom planejamento de demanda, poupa recursos ao produtor que podem ser investidos em melhor qualidade de produto, e menor preço ao consumidor a fim de que seja obtido um modelo de previsão viável para evitar sobrestimação de demanda com consequência de desperdício de alimentos, ou subestimação com consequência de docentes ou discentes sem refeições. 

\paragraph*{} Tal problema tem sido impactante e frequente para o restaurante que informa que em alguns
dias no mês passa por sobrestimação e desperdício superior a 100 refeições. 

\section{Definição do problema}
O problema a definir neste trabalho é encontrar um modelo de previsão de demanda através de algoritmos de regressão por aprendizado de máquina.

\section{Justificativas}
As atuais abordagens de previsão do restaurante universitário, que envolvem média aritmética simples e dedução subjetiva são falhos por não serem calculados tendo uma visão ampla de todo um histórico de dados de grande amostragem, e também não cruzam informações diversas como dados climáticos, dados de calendário anual, feriados próximos, entre outros.

\section{Objetivos}
\subsection{Objetivo geral}
Construir um modelo utilizando uma Rede Neural Artificial para a previsão da demanda de
refeições do restaurante universitário do ICT-UNIFESP com menos de 10% de erro.
\subsection{Objetivos específicos}
\begin{itemize}
\item Construir modelos utilizando três algoritmos de regressão; 
\item Realizar análise estatística da qualidade dos modelos;
\end{itemize}

\section{Metodologia}
\paragraph*{Análise de regressão}

Por se tratar de um trabalho de previsão de demanda,  este trabalho irá realizar a coleta e tratamento dos dados de consumo do restaurante universitário da Unifesp, a coleta de dados climáticos que influenciam em tal demanda, calcular os dados de calendário derivados das datas de coleta, como por exemplo obter o dia da semana de acordo com a data, estação do ano, número do semestre (se par ou ímpar), entre outros.

Após a estruturação dos dados é realizado uma análise exploratória fundamentada por \cite{Junior2007} no capítulo 3.3, para previsões de demanda em geral.
O modelo de dados de venda estruturados receberá o acréscimo de dados de variáveis climáticas, como possíveis fatores de influencia no consumo, conforme ocorre no trabalho de previsão de demanda de energia elétrica fundamentado por \cite{Almeida2013} \cite{Ruas2012} e \cite{Silva2010}

Os trabalhos de \cite{Junior2007} e \cite{Silva2010} fundamentam também uma classificação de análises e métodos de previsão de demanda, onde se conclui a utilização dos métodos de Regressão Linear Múltipla, Redes Neurais por Inteligência Artificial, e avaliação do método dos k-vizinhos.

Será aplicado análises estatísticas de regressão linear múltipla fundamentadas por \cite{Clarice2011}

\paragraph*{Estudo de Aprendizado de máquina}
As análises de inteligência artificial, fundamentadas por trabalhos de previsão de demanda em R.U na Universidade Federal de Viçosa, de acordo \cite{Lopes2008} e na Universidade Estadual Paulista Júlio de Mesquita por \cite{Rocha2011}, concluem a aplicação da técnica de redes neurais artificiais com o método de perceptron de múltiplas camadas fundamentado em \cite{Haykin1994}, e o método dos k-vizinhos.

\paragraph*{Meta-análise}

De acordo com \cite{Flavia2014} que realiza um trabalho de previsão de consumo em aves, a técnica de meta-análise, a análise das análises, será ideal para a comparação e discussão dos resultados, com medidas de avaliação de erro absoluto médio (EAM), erro quadrado médio (EQM) e raiz do erro quadrado médio (REQM). Usando a verificação estatística das diferenças nas medidas obtidas com as técnicas por meio do teste de hipotese de Wilcoxon Rank Sum. 

\section{Organização do documento}
Este trabalho está organizado da seguinte forma: o capítulo 2 apresenta os conceitos básicos para o entendimento do trabalho. O capítulo 3 apresenta trabalhos relacionados. O capítulo 4 demonstra o plano de atividades para o TCC II. O capítulo 5 conclui este trabalho.
% ----------------------------------------------------------
% Fundamentação Teórica
% ----------------------------------------------------------
\chapter{Fundamentação Teórica}
    % ----------------------------------------------------------
    % INTRODUÇÃO
    % ----------------------------------------------------------
\section{Introdução}
A previsão da demanda é o fator principal da eficiência de qualquer modelo de processamento do tipo entrada-saída, onde a sua saída deve atender uma demanda não determinística. É necessário prever a demanda para projetar e aperfeiçoar o processamento e a entrada deste modelo. 

O modelo a ser analisado neste trabalho é o comportamento dos consumidores de um restaurante universitário, onde o mesmo precisa projetar sua compra de insumos e alocação de recursos na entrada de seu modelo de negócio, e projetar sua saída, que é a produção de refeições em quantidade numérica e inteira distribuída em função do tempo em dias, para atender um consumo feito por alunos, que não se comporta de maneira determinística, já que este consumo é facultativo aos alunos.

De acordo com o contrato presente entre o restaurante e a universidade, o mesmo deve atender totalmente à demanda do público, sendo multado se caso algum consumidor fique sem alimentação, porém este mesmo contrato não trata refeições que não são consumidas, logo o restaurante deve lidar integralmente o prejuízo de refeições produzidas acima da demanda de consumo.

Tais refeições fornecidas à alunos também são em parte subsidiadas pela universidade. O periodo que se compreende à agosto de 2018 segue um modelo de contrato antigo no qual o restaurante recebe R\$2,50 do aluno e R\$9,14 da universidade compreendendo o total de R\$11,64 por refeição.

De agosto em diante, iniciando no segundo semestre de 2018, a universidade subsidia R\$5,44 da refeição do aluno e o mesmo R\$2,50, o restaurante recebe o total de R\$7,94. Logo esta previsão de demanda corresponde também aos interesses da administração do campus local, que periodicamente deve realizar uma alocação de recursos financeiros para subsidiar todas estas refeições consumidas.

É necessário então entender e descobrir quais elementos influenciam este consumo humano, de que forma estes elementos exercem tal influência e em qual intensidade ela ocorre.

E para explorar tais elementos, é necessário explorar como este consumo ocorreu historicamente a fim de se encontrar padrões de comportamento, que no cenário deste trabalho acontece com informações distribuídas através do tempo. 

Neste capítulo o principal objetivo é definir todos os conceitos que serão necessários para o entendimento total deste trabalho, analisando a forma que os dados são entendidos, e as ferramentas com os quais possam ser trabalhados.

\section{Dados}
Abstraindo o histórico de vendas do restaurante, temos 2 tipos de dados principais. A venda podendo ser uma observação Y, e a data da venda sendo uma observação X. Se comportando de modo que a função de X implica em Y. 

X,Y $\rightarrow Y \simeq f(X) $

A observação principal a ser analisada neste trabalho é o numero de vendas do R.U do ICT - UNIFESP, em valor inteiro, obtido de um dia letivo, em um determinado período sendo este o almoço ou jantar.

O tempo, onde a informação principal se distribui, é em formato de data, e contém apenas 1 valor da venda total de almoços nesta data, e da venda total de jantas nesta data.

\subsection{Dados de Consumo e de Data.}
Os dados históricos de consumo no restaurante foram retirados do atual sistema banco de dados de refeições subsidiadas do Hospital São Paulo, que gerencia os dados dos refeitórios de todos as unidades da Unifesp. É importante ressaltar que tais dados são apenas de funcionários, docentes e discentes da instituição. Eventuais compras de refeições realizadas por visitantes que não possuem vínculo com a instituição não são registradas pelo sistema. 

Outro fato importante é que estes dados registrados no banco, são registrados no momento em que a refeição é realizada, pelos terminal da entrada do refeitório e não no momento que é vendida pelo caixa do refeitório. Caso fossem registrados no momento da venda no caixa, a análise estatística do consumo poderia sofrer um viés errôneo em relação à frequência de pessoas no refeitório. 

Os alunos são os que realizam o maior volume de consumo, e são os que tem um padrão de frequência à universidade de modo estocástico, que tende a ter uma sazonalidade anual de frequência na universidade desde 2017, quando a instituição padronizou a grade horária de disciplinas.  Além de ser o objetivo principal da previsão por serem os únicos que têm as refeições subsidiadas, pagando um custo de apenas R\$2,50 por refeição, onde entende-se que este custo pode ter mais forte influência na decisão do aluno realizar uma refeição em comparação à outro discente ou pessoa com vínculo na instituição. Por isso os dados que serão analisados neste trabalho, serão somente da frequência de alunos, e terão como início o ano de 2017. 

Apenas alguns funcionários autorizados tem acesso ao banco de dados do sistema de refeições da instituição, entre eles o fiscal de contrato do restaurante universitário. Para obter tais dados neste trabalho, foi necessário obter uma autorização com a direção do campus ICT - UNIFESP e em seguida solicitar a exportação dos dados ao fiscal. Os parâmetros dos quais o mesmo consegue realizar a exportação de dados são diversos, porém foi solicitado o formato de valores separados por vírgula que compreendem o ano de 2017 a 2018.

O modelo exportado segue o seguinte da seguinte forma abaixo, contendo o exemplo de 2 datas: 
\begin{algorithm}[H]
"
CONSULTA POR PERÍODO                    ",,,,,,,,,,
"
CAMPUS: SÃO JOSÉ DOS CAMPOS                    ",,,"
RU: TODOS                    ",,,"
VINCULO: ALUNO                    ",,"
PERÍODO DE: 01/01/2017 A 31/10/2018                        ",,
DATA,VENDAS CAFÉ,VENDAS ALMOÇO,VENDAS JANTAR,VENDAS REFEIÇÃO*,TOTAL VENDAS,ENTR. CAFÉ,ENTR. ALMOÇO,ENTR. JANTAR,TOTAL ENTR. REFEIÇÃO*,TOTAL ENTRADA
(31/10/2018),0,395,0,395,395,0,362,0,362,362
(30/10/2018),0,667,0,667,667,0,437,256,693,693
\end{algorithm}

Logo, no exemplo da segunda linha deste modelo de dados, serão analisados o primeiro valor "(30/10/2018)" como a primeira variável de entrada que corresponde à data, o oitavo valor "437" que corresponde à variável de saída do total de almoço, e ao nono valor "256" que corresponde à variável de saída do total de janta.

Em todo os bancos de dados gerenciados pelo setor de tecnologia da informação da Unifesp, do campus ICT, o mesmo informou que foi constam 428620 refeições subsidiadas no banco de dados do sistema antigo, no período de 2011 à 2016, e 111454 refeições no período de 2017 a 01/08/2018 que fecham o modelo de contrato antigo. O calculo do custos médios cobrados pelo restaurante de R\$9,14 e R\$2,50 pelo aluno, estima que ambos gastaram respectivamente R\$4.936.276,36 e R\$1.350.185,00 enquanto os restaurantes que estivem em contrato nesse período com a universidade, tiveram um faturamento bruto acumulado de R\$6286461,36

Já neste atual 2º semestre de 2018, iniciando em 01/08/2018 à 31/10/2018, o valor total subsidiado pela universidade é de R\$179019,52, R\$82270,00 pelos alunos e o faturamento bruto acumulado pelo restaurante é de R\$261289,52

É importante observar que na implementação computacional das análises de predição, o campo de data é utilizado para o controle da informação e união dos registros climáticos e de consumo, além ser utilizado para obtenção de dados derivados do tempo como o dia da semana, o período do semestre e a estação do ano. Já os campos de número total de refeições consumidas, devem constar como formatos quantitativos nas estruturas de dados dos algoritmos a serem implementados.

Informações derivadas da data não necessitam de coleta, como por exemplo dia da semana, pois podem ser obtidas por métodos de cálculos, sabendo-se apenas um dia da semana correspondente à um dia do ano, ou utilizando bibliotecas computacionais, no caso a biblioteca datetime para a linguagem python.

Para a obtenção de feriados e recessos do calendário acadêmico de um ano determinado, a informação deve ser obtida através do link oficial da instituição, http://www.unifesp.br/reitoria/prograd/pro-reitoria-de-graduacao/informacoes-institucionais/calendario-academico. A única forma de exportação desta informação é em pdf e em formato não tratável facilmente por algoritmos. Então os dias de recesso respectivos ao campus ICT-UNIFESP devem ser carregados manualmente nos códigos a serem implementados.

\subsection{Dados Climáticos}
Também serão analisadas variáveis climáticas juntos com os dados de consumo, de forma a analisar a influência de fatores externos como temperatura média ambiente e precipitação. Tais dados podem ser obtidos de forma gratuita pelo BDMEP - Banco de Dados Meteorológicos para Ensino e Pesquisa, pertencente à instituição pública INMET - Instituto Nacional de Meteorologia, pertencente ao MINISTÉRIO DA AGRICULTURA, PECUÁRIA E ABASTECIMENTO do Governo Brasileiro. 

É necessário um cadastro no site http://www.inmet.gov.br/portal/index.php?r=bdmep/bdmep para a obtenção dos dados. 

A instituição contêm dados registrados de forma digital desde 1961 no país inteiro, os dados históricos referentes a períodos anteriores a 1961 ainda não estão em forma digital e, portanto, estão indisponíveis no BDMEP.

O BDMEP pode fornecer os registros de Precipitação(mm), Temp Máxima(ºC), Temp Mínima(ºC), Insolação(horas)
Evaporação do Piche(mm), Temperatura Compensada Média(ºC), Umidade Relativa Média(\%), Velocidade Vento Média(mps).

Importante ressaltar que o BDMEP leva 90 dias para registrar cada nova data.
Os dados utilizados para este trabalho foram da estação 83781 - São Paulo MIR de Santa - SP.
Observa-se que para futuros trabalhos no atual campus ICT-UNIFESP, a estação TAUBATE - SP 83794, que no momento tem o banco de dados vazio, poderá ser utilizada.

Serão analisadas a temperatura máxima do dia que tem registro em momento próximo à frequência de refeições no horário de almoço, a temperatura mínima do dia que está próxima à janta, e a umidade relativa média que pode indicar presença de chuva no dia, que por sua vez causa diversos impactos ao público da região como a própria frequência do mesmo na instituição, principalmente no fator logístico. 

Os dados se apresentam da seguinte forma: 
\begin{algorithm}[H]
Estacao;Data;Hora;Precipitacao;TempMaxima;TempMinima;Insolacao;Evaporacao Piche;Temp Comp Media;Umidade Relativa Media;Velocidade do Vento Media;
83781;01/01/2017;0000;;31.7;;8.5;6.1;25.2;66;3.466667;
83781;01/01/2017;1200;0;;18.1;;;;;;
83781;02/01/2017;0000;;31.3;;5.6;6.3;23.96;83;2.9;
83781;02/01/2017;1200;16;;18.1;;;;;;
83781;03/01/2017;0000;;31.3;;7;4.5;23.1;87;3;
83781;03/01/2017;1200;3.7;;19.2;;;;;;
\end{algorithm}

Observa-se que cada data contem 2 linhas, sendo a primeira linha com o terceiro campo de valor "0000" apresentando a temperatura máxima do dia no quinto campo, e a segunda linha com o terceiro campo de valor "1200" contendo a temperatura mínima do dia no sexto campo.
A umidade relativa media se encontra no décimo campo da primeira linha de cada data.
Todas as datas apresentam sempre o mesmo padrão de 2 linhas.
    % ----------------------------------------------------------
    % ESTATÍSTICA
    % ----------------------------------------------------------
\section{Análises Estatísticas}
\subsection{Análise Exploratória dos Dados}
\cite{Junior2007} Cita que dados coletados em modelos de previsão possuem informações que quando são projetadas graficamente evidenciam comportamentos que em alguns casos podem ser visualizados e generalizados de forma subjetiva pelos gestores dos dados.  
Em todos os casos, a análise exploratória é necessária para selecionar o melhor método de análise que se enquadra neste comportamento.

COLOCAR AQUI O GRÁFICO DAS VENDAS DE 2017 / 2018 E AS TEMPERATURAS MEDIDAS.

Somente a análise exploratória não é o suficiente e realizada nos intervalos ou critérios incorretos pode comprometer seriamente as conclusões do comportamento dos dados, e que por sua vez pode comprometer seriamente a decisão dos gestores responsáveis por estes dados, no cenário de uma previsão de demanda. 
Isto ocorre atualmente no cenário de previsão da demanda de refeições do ICT-UNIFESP, onde a universidade e o estabelecimento que fornece as refeições não tem nenhum modelo de previsão de demanda. 

De acordo com o gestor da atual empresa que fornece refeições no ICT, a análise utilizada para se prever as refeições é observar dentro da semana o dia anterior de consumo. Em variações de 300 para 450 refeições aproximadamente, isso tem provocado um desperdício médio de 150 refeições diárias. Em geral, de acordo com o restaurante, todos os dias o mesmo trabalha com um erro e um descarte de 30\% das refeições que são trazidas e consumidas ao campus. Estima-se então que no período de 2011 - 01/08/2018 os estabelecimentos tenham tido um prejuízo de R\$1.885.938,40, e de 30\% de R\$78.386,85 no atual período de 01/08/2018 - 31/10/2018 totalizando o montante  R\$1.964.325,25. Aproximadamente 2 milhões de reais em prejuízo acumulado desde 2011.

\subsection{Métodos de Previsão} 

\cite{Junior2007} Realiza uma revisão bibliográfica extensa abordando principais métodos de previsão de consumo sazonais, no cenário de uma indústria cosmética. Tais métodos estatísticos de previsão se dividem em 2 ramificações, sendo quantitativos ou qualitativos. Métodos qualitativos fazem um julgamento dos dados expostos sem um sistema de processamento analítico para se produzir novos modelos ou dados, eles são úteis para sistemas de agrupamento, clusterização ou classificação de dados, sem fornecer novas informações numéricas ou modelos preditivos.

Métodos quantitativos que é o foco deste trabalho, são analíticos e se baseiam em um modelo matemático para realizar previsões. 

Para realizar tais previsões os métodos quantitativos necessitam de um histórico de dados, para analisar padrões em seu comportamento e predizer o futuro que irá agir dentro deste padrão.

Estes métodos se ramificam em 2 tipos, as séries temporais e os modelos causais.
\begin{figure}[!ht]
	\center{
		\includegraphics[width=0.65\textwidth]
		{Figuras/JUNIOR2007/metodos-quantitativos.png}
	}
	\caption{Tipos de métodos quantitativos Retirado de \cite{Junior2007}.\label{fig:metodosQuantitativos}}
\end{figure}

\subsection{Métodos de previsão de Demanda}
O autor supracitado também referencia métodos estatísticos especialmente selecionados para uma previsão de demanda, com a atenção de que alguns métodos qualitativos foram criteriosamente selecionados para prever uma demanda industrial, onde geralmente são previstas pelos métodos quantitativos.

\begin{figure}[!ht]
	\center{
		\includegraphics[width=0.65\textwidth]
		{Figuras/JUNIOR2007/metodos-previsao-demanda.png}
	}
	\caption{Tipos de métodos quantitativos Retirado de \cite{Junior2007}.\label{fig:metodosPrevisaoDemanda}}
\end{figure}


\subsection{Meta-análise}
\cite{Flavia2014} cita em sua tese, que resultados e modelos obtidos a partir de análises experimentais em determinado trabalho científico tem a expectativa de confirmação em um padrão de repetição futura, mas que nem sempre tal expectativa é cumprida. Logo é interessante que seja realizado uma análise das análises, ou seja, a comparação de resultados de diferentes métodos obtendo conclusões mais confiáveis e informativas do que o uso único de uma análise.

O comportamento dos dados deste trabalho, apesar de ter uma distribuição de datas, que são em função do tempo e se classificando em um modelo de série temporal, tem tal comportamento impactado por relações causais com outras variáveis como recesso acadêmico, feriados, eventos, precipitações intensas que causam trânsito local e impactam na logística e frequência do público, entre outras variáveis de causas menos aparentes. Portanto será tratado como um modelo de relação causal, implementando o método estatístico de regressão linear, e os métodos de inteligência artificial dos k-vizinhos mais próximos e redes neurais, que serão explicados nos capítulos a se seguir.

\subsection{Séries Temporais}
De acordo com  \cite{Morettin1987} uma série temporal é um conjunto de observações ordenadas em função do tempo, comumente iguais, apresentando uma dependência serial entre elas, sendo um dos objetivos do estudo de séries temporais, analisar e modelar essa dependência. Além disso, séries temporais são analisadas pelos seus principais movimentos, como tendência, sazonalidade e a componente aleatória, sendo a tendencia e sazonalidade as propriedades que mais ganham destaque em pesquisas de previsão de demanda de carga elétrica. Os meios mais comuns de se analisar a componente sazonal são o Método de Regressão (paramétrico), Método de Médias Móveis (não paramétrico), e Método de Diferença Sazonal (sazonalidade determinística).  Neste trabalho, a distribuição dos dados do Restaurante se dá de forma paramétrica, ou seja, as vendas do restaurante se distribuem em função do tempo e com influências de parâmetros como o dia da semana por exemplo, sendo ideal as análises de regressão. 

\cite{Almeida2013} também cita que para realizar a previsão de uma determinada série temporal é possível utilizar diferentes métodos. Pode-se classificá-los basicamente entre métodos estatísticos e baseados em inteligência artificial.
Dentre os métodos estatísticos destacam-se [4]:
- Regressão Linear Múltipla;
- Alisamento Exponencial;
- Média Móvel Integrada Auto-Regressiva (ARIMA)
Entre os métodos baseados em inteligência artificial tem-se:
- Redes Neurais Artificiais;
- Lógica Fuzzy.

\subsection{Componentes Temporais}

É importante observar que os métodos de previsão em séries temporais buscam uma redução da série temporal à um modelo estacionário e à decomposição da série em componentes de tendência, ciclo, sazonalidade e termo aleatório. A tendencia se entende pelo movimento persistente dos dados em uma direção, o ciclo indica o movimento oscilatório desta tendência, sazonalidade indica comportamento regular assumido de forma repetitiva e o termo aleatório se dá por movimentos irregulares presentes na série.
 
As técnicas de previsão para evidências sazonais usam métodos de regressão, que pode ser observados nos trabalhos de previsão de consumo de energia elétrica realizados por \cite{Almeida2013}, \cite{Ruas2012}, \cite{Silva2010} e de previsão de demanda de produtos cosméticos em \cite{Junior2007}

O cenário deste estudo a frequência de alunos dentro do ICT - UNIFESP e consequentemente dentro do R.U tem sazonalidade anual a partir do momento em que a grade de disciplinas foi fixada também com sazonalidade anual. 

\begin{figure}[!ht]
	\center{
		\includegraphics[width=0.65\textwidth]
		{Figuras/RUAS2012/01-DADOS-DE-DEMANDA-EM-FUNCAO-DO-HORARIO.png}
	}
	\caption{DADOS DE DEMANDA DE SAZONALIDADE DIÁRIA, EM FUNÇÃO DO HORÁRIO. Retirado de \cite{RUAS2012}.\label{fig:seriesTemporais}}
\end{figure}

\subsection{Regressão Linear Múltipla}
A teoria da regressão se iniciou no século XIX com Galton, que analisou a altura dos pais e dos filhos (Xi e Yi) e buscou a influencia da altura do pai no filho, notando que a altura do filho tendia à media da altura do pai, e chamou esta técnica de regressão pois existe uma tendência dos dados regredirem à média.

\cite{Clarice2011} Informa que em estudos de regressão aplica-se o método relacionar uma variável aleatória Y com uma variável X, e no caso da regressão múltipla com múltiplas variáveis X, representando causas diferentes que se combinam para a ocorrência de um valor Y. 

No exemplo abaixo de regressão linear simples, podemos analisar o total de venda Y de um dia X, sendo relacionado nas variações (B0, B1, ..., Bk) de datas de X.

Ou seja, X,Y $\rightarrow Y \simeq f(X) $
Na regressão simples Y depende apenas de uma variável X, isto é, $Y \simeq f(X, B0, B1, ..., Bk) + Ey$, Sendo $Ey$ um erro aplicado à esta função para se chegar no valor Y. Logo a função $f(.)$ pode ser linear nos parâmetros (B0, B1, ..., Bk), se a derivada da função em relação às derivadas dos parâmetros, corresponderem à $h(X),i$, para i variando de 0 à k. Sendo $h(X),$ dependente apenas de X.

Na regressão múltipla, onde temos múltiplas relações causais X1 à Xk, em ocorrências B0 à Bk, temos a função: $Y \simeq f(X1, X2, ..., Xk, B0, B1, ..., Bk) + Ey$. Também sendo linear nos parâmetros se a derivada da função em relação às múltiplas derivadas dos parâmetros, corresponderem à $h(X1,X2,...,Xk)$, tendo $h(.),$ dependente apenas de X1,X2,...,Xk. Em caso contrário, $f(.)$ é uma função não linear dos parâmetros.

Neste trabalho temos múltiplas relações causais, como dados climáticos e dados derivados da data. Logo será usado um algoritmo de regressão linear múltipla. Para realizar a regressão e realizar o teste das derivadas para conferir a linearidade, dos mesmos. Em caso de testes falhos, o modelo de parâmetros será reduzido ou adaptado, para a aprovação novamente no algoritmo, até a obtenção de um modelo final para validação.

O modelo inicial de dados, para o exemplo das 2 primeiras datas letivas de 2018, seguem com os dados de entrada de consumo e em seguida os dados de entrada de clima: 
\begin{algorithm}
DATA,VENDAS CAFÉ,VENDAS ALMOÇO,VENDAS JANTAR,VENDAS REFEIÇÃO*,TOTAL VENDAS,ENTR. CAFÉ,ENTR. ALMOÇO,ENTR. JANTAR,TOTAL ENTR. REFEIÇÃO*,TOTAL ENTRADA;
(26/02/2018),0,930,10,940,940,0,446,190,636,636;
(27/02/2018),0,640,13,653,653,0,470,237,707,707;   
\end{algorithm}

\begin{algorithm}
83781;26/02/2018;0000;;27.2;;1.1;2.9;22.2;88;1.6;
83781;26/02/2018;1200;0;;19.6;;;;;;
83781;27/02/2018;0000;;28.5;;3.6;1.1;22.92;84;2.5;
83781;27/02/2018;1200;73.1;;19;;;;;;   
\end{algorithm}

Os dados de consumo no modelo reduzido de previsão de almoço, fica somente com a data da venda, e o total de refeições realizadas no almoço.

\begin{algorithm}
DATA,ENTR. ALMOÇO;
(26/02/2018),446;
(27/02/2018),470;   
\end{algorithm}

Na tabela de dados climáticos, a redução ocorre para o valor da data, na linha 0000, contendo o valor de temperatura máxima, e precipitação.

\begin{algorithm}
26/02/2018;27.2;88;
27/02/2018;28.5;84;
\end{algorithm}

A informação de data, irá gerar informações derivadas como o valor em forma de fator do dia da semana (obtido através de métodos que retornam o dia da semana, dado uma data, numero do semestre sendo 1 para o primeiro, e 2 para o segundo.

O dia da semana seguirá o modelo binário de acordo com os trabalho de previsão de demanda em R.U realizados por \cite{Lopes2008} e \cite{Rocha2011}.

\begin{figure}[!ht]
	\center{
		\includegraphics[width=0.65\textwidth]
		{Figuras/LOPES/01-ENTRADAS-DIA-SEMANA.png}
	}
	\caption{Entradas de dia da semana em cofatores. Retirado de \cite{Lopes2008}.\label{fig:entradasSemanais}}
\end{figure}

A tabela final resultará em:
A informação de data constará no primeiro campo da estrutura mas será ignorada pelo código, que iniciará a leitura da primeira variável Xi no campo de Vendas almoço neste exemplo.

\begin{table}[!ht]
	\centering
		\caption{Tabela de dados para regressão múltipla.}	\label{tab:regressaoMultipla}
		\begin{tabular}{|c|c|c|c|c|c|c|c|c|c|c|c|c|c|c}
			\hline  \textbf{Data} &	\textbf{Vendas almoço} & \textbf{Temperatura(C)} & \textbf{Precipitação(\%)} & \textbf{Segunda} & \textbf{Terça} & \textbf{Quarta} &\textbf{Quinta} & \textbf{Sexta} & \textbf{Primavera} & \textbf{Verão} & \textbf{Outono} & \textbf{Inverno} & \textbf{1o Semestre} & \textbf{2o Semestre}\\
			\hline 26/02/2018 & 446	& 27.2	& 88 & 1 & 0 & 0 & 0 & 0 & 0 & 1 & 0 & 0 & 1 & 0\\
			\hline 27/02/2018 & 470	& 28.5	& 84 & 0 & 1 & 0 & 0 & 0 & 0 & 1 & 0 & 0 & 1 & 0\\
		\end{tabular}
\end{table}

\paragraph{Algoritmo de Regressão Linear Multipla}
\begin{algorithm}[H]
   \SetAlgoLined
   \Entrada{$T, A$} 
%  \Saida{Número esperado de nodos atingidos}
   \Inicio{
    $S \leftarrow new  Stops()$\ \\
    $M \leftarrow new  Moves()$\ \\
    \Para{$cada \,\, trajetória \,\, t \in T$}{
   	 $i \leftarrow 0$\ \\
     $previousStop \leftarrow null$\ \\
      \Enqto{$i \le size(t)$}{
      	\Se{$\exists\,\,(Rc,\Delta c) \in A \mid geometria(t[i]) \cap Rc$}{
        	$enterTime \leftarrow time(t[i])$\ \\
            $i++$\ \\
        \Enqto{$geometria(t[i]) \cap Rc)$}{
        	$i++$\
        }
        $i--$\ \,\, , \,\,
        $leaveTime \leftarrow time(t[i])$\ \\
        \Se{$leaveTime - enterTime \ge \Delta c$}{
        	$stop \leftarrow (t,Rc,enterTime,leaveTime)$\ \\
            $S.add(stop)$\ \\
            $move \leftarrow (t,previousStop,stop,previousStop.leaveTime,enterTime)$\ \\
            $M.add(move)$\ \\
            $previousStop \leftarrow stop$\
         }
        }
        $i++$\ \,\, ,\,\,
        $j \leftarrow 1$\ \\
        \Enqto{$(i+j \le size(t))\,\, and\,\, (t[i+j]-t[i] < min\,\, \Delta c\, (A))$}{
        	$j++$\
        }
        \Se{$\not\exists \,\,(Rc,\Delta c) \in A \mid geometria(t[i+j-1]) \,\, \cap Rc$}{
        $i \leftarrow i+j$\
        }
   	  }
      \Se{$ t[i-1]\,\, not \in previousStop$}{
      $move \leftarrow (t,previousStop,null,previousStop.leaveTime,time(t[i-1]))$\ \\
      $M.add(move)$\
      }
    }
   }
   \label{algoritmo1}
   \caption{\textsc{Pseudocódigo SMoT}}
\end{algorithm}

    % ----------------------------------------------------------
    % INTELIGENCIA ARTIFICIAL
    % ----------------------------------------------------------
\section{Inteligencia Artificial}
*MOTIVAÇÃO DOS ESTUDOS DE INTELIGÊNCIA ARTIFICIAL* - texto a se completar.
A inteligência artificial surgiu então com o objetivo de reproduzir o comportamento de aprendizado humano e sua capacidade de aprendizado por meio de percepção do ambiente real, generalização e tomada de decisões em novos ambientes de ações que se diferenciam no contexto do ambiente aprendizado porém se assemelham aos parâmetros do ambiente de aprendizado. 

\subsection{Neurônio Artificial}
*bioinspiração do neurônio artificial e abstração para modelo matemático.
O neurônio artificial é a estrutura computacional modelada com inspiração no cérebro humano, que usa elementos de processamento  e calculo semelhante à regressão, interconectados em infinitos modelos e combinações de conexões possíveis.
\cite{Haykin1994} demonstra que este modelo matemático de processador foi desenvolvido com estrutura semelhante do neurônio biológico e de funcionamento baseado no comportamento do sistema nervoso.
Logo para 

Entende-se então que o neurônio artificial é uma função matemática que processa as informações provenientes de variáveis preditivas com suas respectivas ponderações. 

De acordo com a figura , o neurônio tem em sua composição: 
\begin{itemize}
\item  Um conjunto de n conexões de entrada $ (x1,x2,...,xn)$ que correspondem aos dendritos, cujas ligações que simulam as conexões nervosas entre neurônios são realizadas por meio de elementos chamados de peso $(w1,w2,...,wn)$ simulando as sinapses.
\item Uma função soma $\sum $ que processa as sinapses captadas pelas entradas;
\item uma função de ativação $ \delta $ que limita o intervalo do sinal de saída $(y)$ a um valor normalizado.]
\end{itemize}

Neste trabalho então, a estrutura do neurônio artificial irá receber o conjunto de valores de entrada da tabela de dados de predição como entradas $ (x1,x2,...,xn)$ para a produção de uma saída única $ y(.)$ que será o dado predito de venda de almoço, ou de janta.

(inserir equação da soma) / flavia página 29.

\section{Redes Neurais Artificiais}
As RNA são sistemas bioinspirados que utilizam neurônios artificiais em modelos computacionais para a classificações de padrões, simulação de atividades humanas, agrupamento de dados, previsões temporais, entre outras aplicações. As redes neurais devem formar um modelo de agente inteligente que conforme \cite{Haykin1994} devem ser capazes de realizar a tarefa de aprendizado e generalização de um problema. 

O aprendizado é realizado através de um caso real conhecido, no caso deste trabalho o conjunto de dados de venda coletados.
Assim como nas análises estatísticas, as redes neurais podem realizar trabalhos de modo quantitativo e qualitativo, e de acordo com a seção de análise exploratória dos dados, as redes neurais analisadas serão de métodos qualitativos.


\subsection{Previsão de demanda de R.U com rede perceptron múltiplas camadas com back-propagation}
* As RNA de múltiplas camadas resolve o problema de linearidade que o perceptron de única camada e a regressão linear de múltiplas variáveis não consegue realizar. 
*Mais detalhes sobre multiplas camadas.

Em \cite{Lopes2008} a rede neural perceptron de múltiplas camadas é utilizada para tratar a previsão de demanda do R.U da UFV, utilizando apenas como variáveis quantitativas as 5 ultimas observações anteriores ao dia a se analisar, e como variáveis de qualitativas o dia da semana variando de segunda a sexta, conforme modelo já detalhado no método de regressão múltipla.
\begin{figure}[!ht]
	\center{
		\includegraphics[width=0.65\textwidth]
		{Figuras/LOPES/rnaLopes.png}
	}
	\caption{Rede Neural Perceptron de Múltiplas Camadas. Retirado de \cite{Lopes2008}.\label{fig:Rna-Perceptron-MultiLayer}}
\end{figure}

Em \cite{Rocha2011} o trabalho realizado com neurônios artificiais para prever a demanda do R.U da Unesp, envolve apenas uma única camada de entrada e uma segunda camada para saída. 
\begin{figure}[!ht]
	\center{
		\includegraphics[width=0.65\textwidth]
		{Figuras/ROCHA/modeloRnaRocha.png}
	}
	\caption{Rede Neural Perceptron de Múltiplas Camadas. Retirado de  \cite{Rocha2011} \label{fig:rnaRocha}}
\end{figure}
Algorit
O método deste trabalho


\chapter{Trabalhos relacionados}
Entre os últimos artigos publicados em 2015 sobre enriquecimento de trajetórias está o trabalho de \cite{sublime2015}, que tem características específicas que permitem a extração de informações semântica sobre dados geográficos de imagens de alta resolução, esse tipo de extração de semântica é muito usado quando os dados não são obtidos via objetos móveis, já que se obtêm a semântica a partir de dados. Porém com estes dados, são gerados mapas atualizados que são proveitosos para aplicar semântica em trajetórias brutas de objetos móveis. Os estudos em torno de trajetórias semânticas e enriquecimento semântico têm tido um grande avanço nos últimos anos.\\
\indent No trabalho de \cite{alvares2007}, o enriquecimento de trajetórias é feito com informações geográficas para simplificar consultas que são complexas, então é proposto um pré-processamento de modelo de dados para adicionar informações semânticas as trajetórias.\\
\indent Em \cite{laube2005finding}, é introduzido o estudo de comportamento de movimentos de trajetórias, onde o autor explica como se obtêm cada tipo de movimento o que vem a ser proveitoso para este trabalho, o padrão de encontro que é exemplificado no livro é o mais complexo de se calcular, por que temos que medir o intervalo de tempo de todas as trajetórias que passam em um determinado raio \textit{R}.\\
\indent Em \cite{Bogorny2012} uma trajetória bruta pode ser enriquecida com diferentes informações semânticas de acordo com o contexto da aplicação e o problema que o usuário pretende resolver.\\
\indent Neste trabalho é utilizado o método de enriquecimento de trajetórias onde usaremos informações geográficas para conflitar com as trajetórias e dizer se são pontos de \textit{stops} através do algoritmo de \textit{stops and moves}, tendo esses dados enriquecidos semanticamente vai ser implementado um algoritmo que detecta encontros em trajetórias a partir de um raio dado e um tempo de permanência mínimo em cada \textit{stop}.

\chapter{Plano de atividades para o TCC II}
Para a continuação deste trabalho o cronograma abaixo deve ser seguido.

\begin{table}[!ht]
	\centering
		\caption{Plano de atividades para o TCC II}	\label{tab:plano}
		\begin{tabular}{|c|c|c|c|c|c|c|}
			\hline  \textbf{Atividades} &	\textbf{Março} &	\textbf{Abril} & \textbf{Maio} & \textbf{Junho} & \textbf{Julho} \\
			\hline 1 & $\checkmark$ 	& & & & 	\\
			\hline 2 & $\checkmark$	& & & & 	\\
			\hline 3 & 	&$\checkmark$ &$\checkmark$ & & 	\\
			\hline 4 & 	&$\checkmark$ &$\checkmark$ &$\checkmark$ & 	\\
			\hline 5 & 	&$\checkmark$ &$\checkmark$ &$\checkmark$ & 	\\
            \hline 6 &  & & & &$\checkmark$ 	\\
			\hline 
		\end{tabular}
\end{table}
\begin{enumerate}
\item Busca de uma base de dados correta para a aplicação.
\item Evidenciar o contexto semântico a ser aplicado.
\item Implementar o algoritmo SMoT.
\item Implementar o algoritmo que detecta encontros.
\item Análise e comparação dos resultados.
\item Documentação total do trabalho.
\item Apresentação do trabalho.
\end{enumerate}

\chapter{Conclusão}
A forma de enriquecer semânticas de objetos móveis, é uma área de pesquisa que cresce cada vez mais, e a produção de dispositivos móveis cresce de uma forma muito agressiva no mercado resultando em grandes volumes de dados geográficos, porém aonde aplicar todos os métodos que vimos neste trabalho e obter resultados significativos, isso ainda está em sua infância. \\
\indent	Em aplicações onde se usam dados gerados por dispositivos móveis, deve ser feito um estudo muito bem-feito para enriquecer suas trajetórias, onde o contexto da aplicação depende muito na execução do algoritmo para obter excelentes resultados.\\
\indent	A proposta deste trabalho é trabalhar com dados fornecidos de trajetórias brutas e aplicar o método SMoT para enriquecê-las semanticamente e posteriormente analisar essas rotas e indicar em quais pontos houve encontros, independente de qual objeto gerou tais rotas.

%\begin{itemize}
%\item Contextualização e Motivação; 
%\item Definição do problema; 
%\item Justificativas;
%\item Objetivos:  Geral e específicos;
%\item Metodologia e
%\item Organização do documento.
%\end{itemize}

%\subsection{Sobre os Títulos e Capítulos}

%As demais subdivisões do texto (seções, subseções, etc) ... 

%\subsubsection{Título de Subseção}
%Veja aqui um exemplo de citaçao direta \cite{memoir}.


% ----------------------------------------------------------
% Capitulo com exemplos de comandos inseridos de arquivo externo 
% ----------------------------------------------------------

% ---
% Capitulo de revisão de literatura
% ---
%\chapter{Revisão Bibliográfica}

% ---
%\section{Introdução}
% ---

% ---
% primeiro capitulo de Resultados
% ---
%\chapter{Resultados}

% ---
% Finaliza a parte no bookmark do PDF, para que se inicie o bookmark na raiz
% ---
\bookmarksetup{startatroot}% 
% ---

% ---
% Conclusão
% ---

% ----------------------------------------------------------
% ELEMENTOS PÓS-TEXTUAIS
% ----------------------------------------------------------
%\postextual


% ----------------------------------------------------------
% Referências bibliográficas
% ----------------------------------------------------------
%\bibliographystyle{plain}
\bibliography{references}

% ----------------------------------------------------------
% Glossário
% ----------------------------------------------------------
%
% Consulte o manual da classe abntex2 para orientações sobre o glossário.
%
%\glossary

% ----------------------------------------------------------
% Apêndices
% ----------------------------------------------------------

% ---
% Inicia os apêndices
% ---
%\begin{apendicesenv}

% Imprime uma página indicando o início dos apêndices
%\partapendices

% ----------------------------------------------------------
%\chapter{Título de Apêndice}
% ----------------------------------------------------------


% ----------------------------------------------------------
%\chapter{Título do Apêndice}
% ----------------------------------------------------------


%\end{apendicesenv}
% ---


% ----------------------------------------------------------
% Anexos
% ----------------------------------------------------------

% ---
% Inicia os anexos
% ---
%\begin{anexosenv}

% Imprime uma página indicando o início dos anexos
%\partanexos

% ---
%\chapter{Título do Anexo}
% ---

%\end{anexosenv}

\end{document}